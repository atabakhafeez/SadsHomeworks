\documentclass[11pt, oneside]{article}   	% use "amsart" instead of "article" for AMSLaTeX format
\usepackage{geometry}                		% See geometry.pdf to learn the layout options. There are lots.
\geometry{letterpaper}                   		% ... or a4paper or a5paper or ... 
%\geometry{landscape}                		% Activate for for rotated page geometry
%\usepackage[parfill]{parskip}    		% Activate to begin paragraphs with an empty line rather than an indent
\usepackage{graphicx}				% Use pdf, png, jpg, or eps§ with pdflatex; use eps in DVI mode
								% TeX will automatically convert eps --> pdf in pdflatex		
\usepackage{amssymb}

\title{Sads HW5	}
\author{Atabak Hafeez}
%\date{}							% Activate to display a given date or no date

\begin{document}
\maketitle


\section{Verification: Class Invariants}

We need to argue about the three functions. 

\begin{description}
\item[getSize] 
This does not change the value of any mutable variable and just returns size. Hence, the loop invariant holds here.
\item[push]
If an element is pushed, the size is incremented by 1. Hence the length increases by 1 and the size increases by 1, which maintains the loop invariant.
\item[pop] 
If an element is popped, the size is decremented by 1. Hence, the length decreases by 1 and size decreases by 1, which maintains the loop invariant.
\end{description}

The loop invariant holds for a new instance as well: $length(NIL) = size = 0$

\section{Verification: Class Invariants}

\section{Verification: Pure Functions}

Using induction:

Base case: 
Using zero\_left and zero\_right rule from notes: 
$$ zero + n == m + zero \Rightarrow n = m  $$

Step case:
Inductive hypothesis: $ n + m == m + n $
$$ succ(n) + m == m + succ(n) $$
$$ \Rightarrow succ(n + m) == m + succ(n) $$
By inductive hypothesis $ \Rightarrow succ(m + n) == m + succ(n)$
$$ \Rightarrow m + succ(n) == m + succ(n)$$

Hence, proven.

\section{Proof Assistants}

I installed coq. I used an example given with the installation files. The Factorial defined recursively under /theories/Arith was run using the ide. I have given screenshots for each step of the proof check.

\end{document}  