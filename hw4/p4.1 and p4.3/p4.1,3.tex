\documentclass[11pt, oneside]{article}   	% use "amsart" instead of "article" for AMSLaTeX format
\usepackage{geometry}                		% See geometry.pdf to learn the layout options. There are lots.
\geometry{letterpaper}                   		% ... or a4paper or a5paper or ... 
%\geometry{landscape}                		% Activate for for rotated page geometry
%\usepackage[parfill]{parskip}    		% Activate to begin paragraphs with an empty line rather than an indent
\usepackage{graphicx}				% Use pdf, png, jpg, or eps§ with pdflatex; use eps in DVI mode
								% TeX will automatically convert eps --> pdf in pdflatex		
\usepackage{amssymb}

\title{Sads Hw4}
\author{Atabak Hafeez}
%\date{}							% Activate to display a given date or no date

\begin{document}
\maketitle

\section{Verification}
\subsection{Euclidean Algorithm}
\noindent
\textbf{Function Specification}

Pre-condition:
$$ P(m,n) := m > 0, n > 0 $$

Post-condition:

$$Q(m,n,x) := x == EuclideanAlgorithm(m,n) $$

\noindent
\textbf{Loop invariants}

$$ I = EuclideanAlgorithm(x,y) == EuclideanAlgorithm (m,n) $$

\noindent
\textbf{Termination ordering}

For the while loop in the algorithm, the termination ordering is:

$$ x + y $$

\subsection{Factorial}

\noindent
\textbf{Function Specification}

Pre-condition:

$$ P(n) := n > 0 $$

Post-condition: 

$$ P(n, product) := n! == product $$

\newpage
\noindent
\textbf{Loop invariants}

$$ I := product == (factor - 1)! $$

\noindent
\textbf{Termination ordering}

For the while loop in the algorithm, the termination ordering is:

$$ n - factor $$

\section{Dynamic Logic: Practice}

See screenshots. I have run the example code for factorial which contains two implementations for it with the proof of correctness using why3 - both terminal and gui.

\section{Dynamic Logic: Theory}

For these proofs we need that :

\begin{enumerate}
\item $[P]F$: F holds in all successor state of s reachable by evaluating P. (section 9.4.1 in the notes)
\item Definition 11.3 (Theorem). A pure term t : bool is a theorem if it is true in every state.
\item Definition 11.4 (Soundness). A set of rules is sound if all provable formulas are theorems.
\item The rules for if and while from sections 11.1.2 and 11.1.3 respectively
\end{enumerate}

\subsection{If}

The proof for if is simple. There are two possibilities for the branching. Given that C is pure, there is branching for $\neg C$ and $C$. If $C$, if evaluates to $[t]F$, which is true from number 1 above. Else if $\neg C$, the same hold i.e. $[t']F$ is true. Hence, $[t]F$ and  $[t']F$ are theorems $\Rightarrow$ \textbf{if} is sound.

\subsection{while}


The while rule has three rules which need to shown are theorems. The first rule is just $I$ which is obviously a theorem as it is the loop invariant assumption before the while loop starts. The second rule is the definition of the loop invariant itself. Assume $I$ and $C$, then, $[t]I$ is a theorem by the same reason as above, using number 1 from above. The third rule is also a theorem because at the this point $\neg C$ becomes true and we already assumed $I$, and finally we get $F$ from where we can continue the proof. Hence, \textbf{while} is sound.




























\end{document}  